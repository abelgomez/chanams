\section{Resolución de problemas}
\label{sec:resolucion-problemas}

A continuación, se describen los procedimientos básicos a realizar cuando se disparan los diversos avisos y alarmas del \MEE.

\subsection{En caso de batería baja}
\label{sec:bateria-baja}

\begin{enumeratecompact}

\item \textbf{Reemplace las baterías del \MEE.} Para ello:

\begin{itemizecompact}

\item acceda al \MEE; 
\item retire la cubierta trasera transparente;
\item retire las baterías antiguas; y finalmente,
\item coloque las nuevas siguiendo las indicaciones del portabaterías. 

\end{itemizecompact}

Una vez haya cambiado las baterías, espere a que el \MIE y el \MEE se sincronicen correctamente.  

\begin{itemizecompact}

\item Si los módulos se sincronizan y los avisos sonoros de batería baja se han detenido, el procedimiento ha terminado.

\item Si los módulos se sincronizan pero los avisos sonoros de batería baja siguen sonando, continúe en el paso 2.

\item En caso de no sincronización, siga el procedimiento que se detalla en la sección~\ref{sec:conn-perdida} (\textit{\nameref{sec:conn-perdida}}).

\end{itemizecompact}

\item \textbf{Compruebe que las nuevas baterías tienen el nivel de carga correcto.} Para ello, revise el nivel de batería en la pantalla 2 del menú de información del \MIE pulsando el botón multifunción \circled{I8} (ver sección~\ref{sec:menu-info}, \textit{\nameref{sec:menu-info}}). Tenga en cuenta que:

\begin{itemizecompact}

\item Unas baterías recién cargadas proporcionan típicamente un voltaje superior a los 5,0 V cuando están en perfectas condiciones. Si el valor es menor, las baterías han comenzado a degradarse.

\item Unas baterías en buenas condiciones deben de permanecer la mayor parte del tiempo de uso en un valor alrededor de los 4,8 V -- 4,9 V.

\item Si las baterías se han cargado recientemente, y su voltaje ha caído rápidamente tras poco tiempo se uso, las baterías están degradas. Reemplácelas por un juego nuevo y vuelva al paso 1.

\item Si las baterías muestran valores normales, pero los avisos de batería baja continúan sonando regularmente, continúe en el paso 3.

\end{itemizecompact}

\item \textbf{Compruebe el \emph{Umbral de batería baja} configurado en el \MIE.} Para ello acceda al modo gestión siguiendo el procedimiento detallado en la sección~\ref{sec:acceso-gestion}. Un valor razonable para el \emph{Umbral de batería baja} estará típicamente entre los 4,6 V y 4,8 V aproximadamente (véase la figura~\ref{fig:discharge-curve} para poder estimar los valores adecuados). En todo caso, el valor deberá ser inferior al voltaje que las baterías del \ME reportan la mayor parte de su vida útil.

\begin{itemizecompact}

\item Si necesita ajustar el \emph{Umbral de batería baja}, revise la sección~\ref{sec:params-interior}.

\end{itemizecompact}

\end{enumeratecompact}



\importantbegin{No emplear pilas alcalinas}
\textbf{RECUERDE: el \MEE debe emplear 4 pilas AA recargables de NiMH de 1,2 V}. El empleo de pilas alcalinas puede provocar daños permanentes por sobretensión en el \ME. 
\importantend


\subsection{En caso de depósito lleno}
\label{sec:deposito-lleno}

\begin{enumeratecompact}

\item \textbf{Vacíe el depósito que está siendo monitorizado por el \MEE.} Para ello:

\begin{itemizecompact}

\item Acceda donde se encuentre el \MEE.

\item Retire el sensor de flotador del interior del depósito.

\item Vacíe el depósito en un lugar adecuado.

\item Vuelva a colocar el depósito en su lugar, e inserte de nuevo el flotador en el interior del depósito a una altura adecuada cerca del límite de su capacidad.  

\end{itemizecompact}

\item \textbf{Deje que el \MEE y el \MIE actualicen su estado.} Para ello puede (alternativamente):

\begin{enumeratecompact}
  
\item esperar a que el \MEE envíe el siguiente mensaje de sincronización según el \emph{Tiempo de espera entre mensajes} configurado; o

\item reiniciar el \MEE forzando la sincronización. Para ello, coloque el interruptor de encendido \circled{E2} en la posición \off, espere unos segundos, y vuelva a conectar el \ME moviendo el interruptur de encendido \circled{E2} a la posición~\on. En caso de no sincronización, siga el procedimiento que se detalla en la sección~\ref{sec:conn-perdida} (\textit{\nameref{sec:conn-perdida}}).
  
\end{enumeratecompact}

\item \textbf{Compruebe que la alarma de depósito lleno deja de sonar.} 

\begin{itemizecompact}

\item Si la \emph{alarma de depósito lleno} ha dejado de sonar, el procedimiento ha terminado.

\item Si la \emph{alarma de depósito lleno} continúa sonando aunque el depósito haya sido vaciado. Compruebe que:

\begin{itemizecompact}

\item El sensor de flotador está en posición vertical.

\item El flotador del sensor se encuentra en la posición inferior.

\item El sensor está limpio y el flotador se mueve con facilidad.

\item El cable del sensor de flotador y los conectores están en buenas condiciones y libres de óxido.

\end{itemizecompact}

Una vez hechas estas comprobaciones, vuelva al paso 2.

\end{itemizecompact}

\item \textbf{\color{main}Si tras haber seguido los puntos previos, la \emph{alarma de depósito lleno} continúa sonando,} existe un problema anormal con el sensor de flotador. Desconecte el \CMS y llévelo a reparar.

\end{enumeratecompact}

\subsection{En caso de conexión perdida}
\label{sec:conn-perdida}
En caso de que el \CMS le alerte de que se ha perdido la conexión entre en \MIE y el \MEE, proceda de la siguiente manera:

\begin{enumeratecompact}

\item \textbf{Reinicie el \MIE}. Para ello, pulse el botón de reinicio \circled{I5} ---o alterntivamente, desconecte y reconecte pasados unos segundos la alimentación USB \circled{I6}--- y espere a que el \MI informe en la pantalla LCD \circled{I1} de que se encuentra conectado a una red wifi.

\begin{itemizecompact}

\item Si tras varios intentos el \MI no se conecta a una red wifi, entre en el modo de gestión (sección~\ref{sec:gestion-avanzada}, \textit{\nameref{sec:gestion-avanzada}}) y configure de nuevo el nombre de la red y la contraseña (sección~\ref{sec:config}, \textit{\nameref{sec:config}}).

\end{itemizecompact}

\item \textbf{Reinicie el \MEE y compruebe que dispone de batería}. Para ello, coloque el interruptor de encendido \circled{E2} en la posición \off, espere unos segundos, y vuelva a conectar el \ME moviendo el interruptor de encendido \circled{E2} a la posición~\on. 

\begin{itemizecompact}

\item Si el testigo de actividad \circled{E1} no emite ningún destello, reemplace las pilas del \ME, y vuelva al inicio del paso 2.

\item Si el testigo de actividad \circled{E1} emite un breve destello azul, continúe. 

\end{itemizecompact}


\item \textbf{Compruebe de nuevo los testigos del \MIE pasados unos instantes (máximo 45 segundos)}.

\begin{itemizecompact}

\item Si el testigo de fallo \circled{I2} se ha apagado y el testigo de conexión \circled{I4} ha pasado a color \textbf{verde fijo}, la conexión se ha reestablecido y no se debe realizar ninguna acción más.

\item Si el testigo de fallo \circled{I2} se ha apagado y el testigo de conexión \circled{I4} ha pasado a color \textbf{verde parpadeante}, la conexión se ha reestablecido pero se deben reemplazar las pilas del \ME a la mayor brevedad posible.

\item Si el testigo de fallo \circled{I2} continúa encendido, repita de nuevo los pasos 1 y 2. \textbf{\color{main}Si tras varios intentos} el testigo de fallo \circled{I2} permanece encedido, entre en el modo de gestión del \ME (sección~\ref{sec:gestion-avanzada}, \textit{\nameref{sec:gestion-avanzada}}) y compruebe que tanto el \MI como el \ME están conectados a la misma red wifi con la contraseña correcta, y que los valores \emph{Nombre o IP del host interior} y \emph{Puerto de escucha del host interior} configurados en el \ME se corresponden con \emph{Nombre de host} y \emph{Puerto de escucha} del \MI~---mostrados también en las pantallas 3 (figura~\ref{fig:screen3}) y 4 (figura~\ref{fig:screen4}) del módulo interior.

\end{itemizecompact}

\end{enumeratecompact}

\section{Guía de inicio rápido}
\label{sec:inicio-rapido}

El \CMS requiere una configuración inicial para poder emplearse por primera vez.
Como mínimo, es necesario que se establezca el nombre y la contraseña de la red wifi a la que los módulos se conectarán para que puedan comunicarse entre ellos.
Esta configuración debe realizarse tanto en el \MIE como en el \MEE.

\subsection{Configuración rápida del módulo interior}

El \MIE debe configurarse primero siguiendo los siguiente pasos:

\begin{enumeratecompact}

\item Mientras pulsa el botón multifunción \circled{I8}, conecte la alimentación USB \circled{I6} del \MI. Cuando el testigo de configuración \circled{I3} se encienda de forma fija y en la pantalla LCD \circled{I1} aparezca el texto \emph{Conecta a Config.~Interior} (ver figura~\ref{fig:screen-config}, página \pageref{fig:screen-config}), suelte el botón \circled{I8}.

Si el mensaje \emph{Conecta a Config.~Interior} no aparece en la pantalla LCD, desconecte la alimentación y comience de nuevo. Alternativamente, en lugar de desconectar y conectar la alimentación USB \circled{I6}, puede presionar el botón de reinicio \circled{I5} mientras pulsa el botón multifunción \circled{I8}.

\item Con la ayuda de un dispositivo móvil o un ordenador con conexión wifi, busque las redes wifi disponibles y conecte a la red llamada \emph{Config.~Interior} (figura~\ref{fig:interior-select-wifi}, página~\pageref{fig:interior-select-wifi}).

\item En caso de que su dispositivo no le rediriga automáticamente a la página de configuración del \MI, despliegue el área de notificaciones de su dispositivo, y pulse la opción \emph{Iniciar sesión en red Wi-Fi ``Config.~Interior''} (figura~\ref{fig:interior-captive-portal-notification}, pá\-gi\-na~\pageref{fig:interior-captive-portal-notification}).

\item Seleccione la opción \emph{Configurar} (figura~\ref{fig:interior-menu}, página~\pageref{fig:interior-menu}).

\item Tras unos segundos, se mostrarán la lista de redes wifi detectadas por el \MI. Pulsando sobre el nombre de una red, se rellenará el campo \emph{SSID} (ver figura~\ref{fig:interior-config}). A continuación, escriba la contraseña de la red wifi seleccionada en el campo \emph{password}. Aplique los cambios pulsando el botón \emph{Guardar} al final de la página de configuración.

\item Se mostrará el mensaje \emph{Configuración guardada. Conectando a la red\ldots} El módulo se reiniciará en unos segundos. El \MI intentará conectarse automáticamente a la red configurada (ver figura~\ref{fig:screen-conn-process}, página~\pageref{fig:screen-conn-process}) y se quedará a la espera de que el \ME esté listo (el testigo de configuración \circled{I3} parpadeará). En caso de error (figura~\ref{fig:screen-conn-failed}, página~\pageref{fig:screen-conn-failed}), pruebe de nuevo desde el paso 1.

\end{enumeratecompact}


\subsection{Configuración rápida del módulo exterior}

El \MEE se debe configurar en segundo lugar siguiendo los pasos a continuación. El proceso de configuración del \MEE es similar al del \MIE:

\begin{enumeratecompact}

\item Asegúrese de que el \ME está apagado (interruptor de encendido \circled{E2} en posición \off).

\item Mientras pulsa el botón de configuración \circled{E5} con ayuda de alguna herramienta de punta estrecha y alargada, conecte el \ME moviendo el interruptor de encendido \circled{E2} a la posición \on. El testigo de actividad \circled{E1} se encenderá momentáneamente. 

\item Con la ayuda de un dispositivo móvil o un ordenador con conexión wifi, busque las redes wifi disponibles y conecte a la red llamada \emph{Config.~Exterior} (figura~\ref{fig:exterior-select-wifi}).

\item En caso de que su dispositivo no le rediriga automáticamente a la página de configuración del \ME, despliegue el área de notificaciones de su dispositivo y pulse la opción \emph{Iniciar sesión en red Wi-Fi ``Config.~Exterior''} (figura~\ref{fig:exterior-captive-portal-notification}).

\item Seleccione la opción \emph{Configurar} (ver figura~\ref{fig:exterior-menu}).

\item Tras unos segundos, se mostrarán las redes wifi detectadas por el \ME. Pulsando sobre el nombre de una red, se rellenará el campo \emph{SSID} (ver figura~\ref{fig:exterior-config}). A continuación, escriba la contraseña de la red wifi seleccionada en el campo \emph{password}. Aplique los cambios pulsando el botón \emph{Guardar} al final de la página de configuración.

\item Se mostrará el mensaje \emph{Configuración guardada. Conectando a la red\ldots} El sistema se reiniciará en unos segundos. El \ME se conectará automáticamente a la red configurada y se sincronizará con el \MI: el testigo de configuración \circled{I3} dejará de papadear y se apagará, y en su lugar, se encenderá el testigo de conexión \circled{I4}. Este proceso puede tardar hasta 45 segundos. En caso de error, pruebe de nuevo desde el paso 1.

\end{enumeratecompact}



\tipbegin{Configuración avanzada}
El \CMS permite configurar muchos otros parámetros de funcionamiento. Revise la sección~\ref{sec:gestion-avanzada} para ver las opciones de gestión y configuración avanzadas.
\tipend


